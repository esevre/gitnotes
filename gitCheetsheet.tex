\documentclass[a4paper]{spie} %% Use article or spie for the document class
\usepackage[latin1]{inputenc}
\usepackage{amsmath}
\usepackage{amsfonts}
\usepackage{amssymb}

%% This is for SI Units:
\usepackage{siunitx}

%% For graphics:
\usepackage{graphicx}


\usepackage{url}
\usepackage{xcolor}
\usepackage{listings}
\lstset{basicstyle=\ttfamily,
  showstringspaces=false,
  commentstyle=\color{red},
  keywordstyle=\color{black}
}
\author{Erik O. D. Sevre}
\title{Git'in Started with Git}

\authorinfo{This document is created by Erik Sevre: E-mail: esevre@gmail.com}

\begin{document}
\maketitle

\section*{Git'in Started with Git}

There are two ways to start a Git project. You can (1) check it out from a repository, or you can (2) create your own. I will look at both approaches. I will start by creating my own, then describe how to check it in to a one of the popular Git repos online.

\subsection*{Creating a local Git Repository}

I will first go through how to do this from the command line. If you have never used the command line please see my "Getting Started with the Terminal" notes.

First I will create a directory called \texttt{mygitrepo} and on my system the directory location looks like:
\begin{lstlisting}
/Users/esevre/Dropbox/Notes/git/mygitrepo
\end{lstlisting}

Currently this directory is empty so I will start by initializing this as a simple git repository.

Add your name and email to the global git config file on your machine.
\begin{lstlisting}[language=bash]
git config --global user.name "Erik Sevre"
git config --global user.email "esevre@gmail.com"
\end{lstlisting}

Or if you want to check the current username and email address in the global config file you can use the following commands:
\begin{lstlisting}[language=bash]
git config --global user.name
git config --global user.email
\end{lstlisting}


\subsection{Init'in a Git Repository}
To create a git repo in our directory described above we simply need to run the following command from the command line in the desired directory
\begin{lstlisting}[language=bash]
git init
\end{lstlisting}

Now typing \texttt{ls -la} allows us to see that there is now a directory called \texttt{.git} hidden. This \texttt{.git} directory will keep track of all our git changes that we monitor along the way.












\end{document}
