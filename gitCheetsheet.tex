\documentclass[a4paper]{spie} %% Use article or spie for the document class
\usepackage[latin1]{inputenc}
\usepackage{amsmath}
\usepackage{amsfonts}
\usepackage{amssymb}

%% This is for SI Units:
\usepackage{siunitx}

%% For graphics:
\usepackage{graphicx}


\usepackage{url}
\usepackage{xcolor}
\usepackage{listings}
\lstset{basicstyle=\ttfamily,
  showstringspaces=false,
  commentstyle=\color{red},
  keywordstyle=\color{black}
}
\author{Erik O. D. Sevre}
\title{Git'in Started with Git}

\authorinfo{This document is created by Erik Sevre: E-mail: esevre@gmail.com}

\begin{document}
\maketitle

\section*{Git'in Started with Git}

There are two ways to start a Git project. You can (1) check it out from a repository, or you can (2) create your own. I will look at both approaches. I will start by creating my own, then describe how to check it in to a one of the popular Git repos online.

\subsection*{Creating a local Git Repository}

I will first go through how to do this from the command line. If you have never used the command line please see my "Getting Started with the Terminal" notes.

First I will create a directory called \texttt{mygitrepo} and on my system the directory location looks like:
\begin{lstlisting}[language=bash]
/Users/esevre/Dropbox/Notes/git/mygitrepo
\end{lstlisting}

Currently this directory is empty so I will start by initializing this as a simple git repository.

Add your name and email to the global git config file on your machine.
\begin{lstlisting}[language=bash]
git config --global user.name "Erik Sevre"
git config --global user.email "esevre@gmail.com"
\end{lstlisting}

Or if you want to check the current username and email address in the global config file you can use the following commands:
\begin{lstlisting}[language=bash]
git config --global user.name
git config --global user.email
\end{lstlisting}


\subsection{Init'in a Git Repository}
To create a git repo in our directory described above we simply need to run the following command from the command line in the desired directory
\begin{lstlisting}[language=bash]
git init
\end{lstlisting}

Now typing \texttt{ls -la} allows us to see that there is now a directory called \texttt{.git} hidden. This \texttt{.git} directory will keep track of all our git changes that we monitor along the way.

Once we have a file in our directory, there are two steps to starting the git repositor. First we need to add the file to the repo, this simply tells the repo to keep track of the file. Then we need to commit our file, this is basically saying "hey git, write down the file the way it appears now!" When we commit we do need to add a comment describing what we are doing with the commit.

So I'm going to add these notes to my git repo, so I copy them to the directory, add them to the git, then commit the file with the following commands:
\begin{lstlisting}[language=bash]
cp ../*.tex .
git add *.tex
git commit -m 'initial setup with starting tex file'
\end{lstlisting}



To ignore files add a file named \texttt{.gitignore} to the top directory of the repo we are working in. Since I am working with \texttt{.tex} files I don't want to keep track of all the \texttt{.log}, \texttt{.aux}, and \texttt{.gz} files generated by my TeX editor, but I will add the final PDF to the repo.
\begin{lstlisting}[language=bash]
git add .gitignore *.pdf
git status
\end{lstlisting}

\texttt{git status} will show us the status of all the changes to date so far



\section*{Commit'in to a Remote Repo}
Here we will look at how to connect to a remote repo like Github or Bitbucket. 

I've created a repo at Github and Bitbucket, so now I need to tell my project about these repos and how to push to them to the respective repo.

We start by looking at what remote repos our project is connected to, we do this with the following command:
\begin{lstlisting}[language=bash]
git remote
\end{lstlisting}

We don't have a remote repo yet so this shouldn't output anything yet. So we will add a repo at Bitbucket and Github. You only need a repo at one of these locations, but I will show you how to manage both repos if that is what you wish to do.


\begin{lstlisting}[language=bash]
git remote add github_origin https://github.com/esevre/gitnotes.git
\end{lstlisting}
\begin{lstlisting}[language=bash]
git remote add bitbucket_origin https://esevre@bitbucket.org/esevre/gitnotes.git
\end{lstlisting}


\section*{Command List}
\begin{lstlisting}[language=bash]
git config --global user.name "Erik Sevre"
git config --global user.email "esevre@gmail.com"
\end{lstlisting}
\begin{lstlisting}[language=bash]
git init
\end{lstlisting}
\begin{lstlisting}[language=bash]
git add file1 file2 file*
git commit -m 'initial setup with starting tex file'
\end{lstlisting}
Adding future files, the \texttt{-a} flag automatically adds modified files to be updated
\begin{lstlisting}[language=bash]
git commit -a -m 'Commiting a future change'
\end{lstlisting}
Add a gitignore file
\begin{lstlisting}[language=bash]
git add .gitignore
git status
\end{lstlisting}
\begin{lstlisting}[language=bash]
git remote
git remote -v
git remote add github_origin https://github.com/esevre/gitnotes.git

\end{lstlisting}




\end{document}
